\documentclass[a4paper,10pt]{article}

\title{Tuto Android - Faire une apli du feu de dieu !!}

\author{TERRIE Corentin \\ CROS Bastien \\ MONNIER Matthias}

\date{\today}


\usepackage[utf8]{inputenc}
\usepackage[french]{babel} 
\usepackage{lmodern} % Pour changer le pack de police
\usepackage{makeidx}
\usepackage{fancyhdr}
\usepackage{graphicx}
\usepackage[lofdepth,lotdepth]{subfig}
\usepackage{float}
\usepackage{hyperref}
%---------------------------------------------------------
\usepackage{listings}
\usepackage{textcomp}
\usepackage{color}
% JAVA en couleur ;-)- -----------------------------------
\lstset{
language=Java,
basicstyle=\scriptsize, % ou ça==> basicstyle=\normalsize,
upquote=true,
aboveskip={1.5\baselineskip},
columns=fullflexible,
showstringspaces=false,
extendedchars=true,
breaklines=true,
showtabs=false,
showspaces=false,
showstringspaces=false,
identifierstyle=\ttfamily,
keywordstyle=\color[rgb]{0,0,1},
commentstyle=\color[rgb]{0.133,0.545,0.133},
stringstyle=\color[rgb]{0.627,0.126,0.941},
breaklines=true,
}
%----------------------------------------------------------

\begin{document}
\include{pagedegarde}
\tableofcontents
\clearpage


%-------------------------------------------------------------------------------


\section{Notion élémentaire Watson} % Intro aux développement Android


%-------------------------------------------------------------------------------------------------------------

\section{Démarrer sous Android}
%install et tout et tout

\subsection{Sous Windows}
% Indiquer la liste des programmes

\subsection{Sous Linux}
% ligne de code nécessaire à l'install

%-------------------------------------------------------------------------------------------------------------

\section{Une première appli - Interface de base}

\subsection{App 1}
% appli basique : par exemple block note du site du zéro
\subsection{Cycle de vie d'une activité}
% 2nd partie : gestion du cycle de vie

%-------------------------------------------------------------------------------------------------------------

\section{Complexification de l'appli - Utiliser plusieurs activités}
% on améliore l'appli en affichant ce que l'on rentre dans le block note dans une seconde
% activité  

%-------------------------------------------------------------------------------------------------------------

\section{Intégration du joystick}
% on part d'une appli qui propose le choix entre joystick et gyro
% on intègre le joystic dans une nouvelle activité

%-------------------------------------------------------------------------------------------------------------

\section{Utilisation des capteurs}
% on intège les gyro dans une autre activité
Pour apprendre à utiliser les capteurs de votre support android, le tutoriel de Matthias Seguy est très bien fait. 
Voici le lien : 

\url{http://mathias-seguy.developpez.com/cours/android/android-capteurs}
%-------------------------------------------------------------------------------------------------------------

\section{Communication avec une STM}

\subsection{Connexion Bluetooth}
% connection BT -> intro a l'api BT, mode de connexion , ordre, etc.
% fais le tour des possibilités de base de l'api
% '-> à metre dans le 1er paragraphe. Ici juste un intro.

	\paragraph{API Bluetooth Android}
	% Besoin si différent de l'intro !!!
	% presenter les différents élements 
	% partie pas mise en forme !!
	Service offert par l'API : \\
		- Scanner les périphériques BT. \\
		- Connaître les différents périphériques appareillés. \\
		- RFCOMM. \\
		- Connection à des périphériques BT. \\
		- Transfert de données via BT. \\
		- Multiple connection. \\\\
	RFCOMM $->$ serial port emulation (RS-232 serial port) $=$ protocole bluetooth pour simuler une connexion série (jusqu'à 6) avec un autre péripherique.\\\\
	Quelques classes offerte par l'API BT : \\
	- \textit{BluetoothAdapter} $=$ Interface physique BT pour interagir avec son la radio BT locale. \\
	- \textit{BluetoothDevice} $=$ Chacun représente un périphériques BT distant. Contient les inforations telles que le nom, l'adresse etc. \\
	- \textit{BluetoothSocket} $=$ (équivalent des TCP Socket) Point de connexion avec un périphériques, permet d'échanger des données. \\
	- \textit{BluetoothClass} $=$ Décrit les caractéristiques d'un périphériques BT distant. \\
	\paragraph{Première approche du Bluetooth : \\}
	Pour découvrir l'utilisation du Bluetooth sur une tablette Android, voici un petits tuto assez complet : \\
	\url{http://www.tutos-android.com/utilisation-bluetooth-application-android} \\
	Il permet d'appréhender les fonctionnalités de base offerte par l'API bluetooth, telle que l'autorisation et l'activation de celui-ci. Il explique aussi comment scanner les périphériques environnant, et parle de la connexion entre périphériques bluetooth, du coté serveur et client.
	
	
	\paragraph{Se connecter à un module Bluetooth}
	
	En sebasant sur l'application proposé par le site suivant : \\
	\url{http://marcel.cremmel.llc.free.fr/Projets/Android_files/Tutorial%20Android%20-%20Communication%20Bluetooth%20SPP.pdf}
	Nous allons maitenant créer une classe service permettant de gérer les connexion Bluetooth.
	\subparagraph{Les services}
	Le service est l'équivalent d'une Activités, mais sans interface graphique. Cela le rends pratique pour exécuter des tâche en arrière-plan. Par exemple, lorsque l'on écoute de la musique sur son téléphone, il est possible de revenir sur le bureau, et de lancer d'autre application sans que la musique s'arrète : c'est la tout l'intérêt des services.
	Pour implémenter un service, il faut créer une nouvelle classe, que l'on nommera \textit{BluetoothService}, et la faire hériter de la classe \textit{service} grâce au mot clef \textit{extends} :
	\begin{lstlisting}
	public class BluetoothService extends Service
	{
			 Ici on implémente les méthodes de la classe
	}
	\end{lstlisting}
	
	
	\paragraph{Gérer les devices}
	% lister périphériqus déjà existant 
	% recherche de nouveau périphériques -> parleer de BroadcastReceiver
	\begin{lstlisting}
		@Override
 
		protected void onDestroy() {
  			super.onDestroy();
  			bluetoothAdapter.cancelDiscovery();
  			unregisterReceiver(bluetoothReceiver);
		}
	\end{lstlisting}
	\paragraph{Connexion à un périphériques}
	
	\subparagraph{Client}
	
	\subparagraph{Serveur}
	
	\subparagraph{Périphérique exterieur}
	% connexion avec un peripherique autre qu'andoid
	

\subsection{Introduction au service}
% introduction au service
	% -> Quoi c'est ?
	% -> Cycle de vie
	% -> la class service
	% -> intérargir avec ??
\subsection{Programme STM32 Nucléo}
% programme de la stm

%-------------------------------------------------------------------------------------------------------------

\section{Foire aux conseils ;-)}
% liste des galères et leurs solutions !

\end{document}
