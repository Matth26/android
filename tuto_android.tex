\documentclass[a4paper,10pt]{article}

\title{Tuto Android - Faire une aplli du feu de dieu !!}

\author{TERRIE Corentin \\ CROS Bastien \\ MONNIER Matthias}

\date{\today}

\usepackage[utf8]{inputenc}
\usepackage[french]{babel} 
\usepackage{lmodern} % Pour changer le pack de police
\usepackage{makeidx}
\usepackage{fancyhdr}
\usepackage{graphicx}
\usepackage[lofdepth,lotdepth]{subfig}
\usepackage{float}
\usepackage{hyperref}
%---------------------------------------------------------
\usepackage{listings}
\usepackage{textcomp}
% JAVA en couleur ;-)- -----------------------------------
%\lstset{
%language=Java,
%basicstyle=\normalsize, % ou ça==> basicstyle=\scriptsize,
%upquote=true,
%aboveskip={1.5\baselineskip},
%columns=fullflexible,
%showstringspaces=false,
%extendedchars=true,
%breaklines=true,
%showtabs=false,
%showspaces=false,
%showstringspaces=false,
%identifierstyle=\ttfamily,
%keywordstyle=\color[rgb]{0,0,1},
%commentstyle=\color[rgb]{0.133,0.545,0.133},
%stringstyle=\color[rgb]{0.627,0.126,0.941},
%}
%----------------------------------------------------------

\begin{document}
\include{pagedegarde}
\tableofcontents
\clearpage


%-------------------------------------------------------------------------------

\section{Notion élémentaire Watson} % Intro aux développement Android


%-------------------------------------------------------------------------------------------------------------

\section{Démarrer sous Android}
%install et tout et tout

\subsection{Sous Windows}
% Indiquer la liste des programmes

\subsection{Sous Linux}
% ligne de code nécessaire à l'install

%-------------------------------------------------------------------------------------------------------------

\section{Une première appli - Interface de base}

\subsection{App 1}
% appli basique : par exemple block note du site du zéro
\subsection{Cycle de vie d'une activité}
% 2nd partie : gestion du cycle de vie

%-------------------------------------------------------------------------------------------------------------

\section{Complexification de l'appli - Utiliser plusieurs activités}
% on améliore l'appli en affichant ce que l'on rentre dans le block note dans une seconde
% activité  

%-------------------------------------------------------------------------------------------------------------

\section{Intégration du joystick}
% on part d'une appli qui propose le choix entre joystick et gyro
% on intègre le joystic dans une nouvelle activité

%-------------------------------------------------------------------------------------------------------------

\section{Utilisation des capteurs}
% on intège les gyro dans une autre activité
Pour apprendre à utiliser les capteurs de votre support android, le tutoriel de Matthias Seguy est très bien fait. 
Voici le lien : 

\url{http://mathias-seguy.developpez.com/cours/android/android-capteurs}
%-------------------------------------------------------------------------------------------------------------

\section{Communication avec une STM}

\subsection{Connexion Bluetooth}
% connection BT -> intro a l'api BT, mode de connexion , ordre, etc.
% fais le tour des possibilités de base de l'api
	\paragraph{API Bluetooth Android}
	% Besoin si différent de l'intro !!!
	% presenter les différents élements 
	\paragraph{Permission}
	% courte intro sur permission
	% implémenter la permission
	\paragraph{Bluetooth Adapter}
	
	\paragraph{Gérer les devices}
	% lister périphériqus déjà existant 
	% recherche de nouveau périphériques -> parleer de BroadcastReceiver
	\paragraph{Connexion à un périphériques}
	
	\subparagraph{Client}
	
	\subparagraph{Serveur}
	
	\subparagraph{Périphérique exterieur}
	% connexion avec un peripherique autre qu'andoid
	

\subsection{Introduction au service}
% introduction au service
	% -> Quoi c'est ?
	% -> Cycle de vie
	% -> la class service
	% -> intérargir avec ??
\subsection{Programme STM32 Nucléo}
% programme de la stm

%-------------------------------------------------------------------------------------------------------------

\section{Foire aux conseils ;-)}
% liste des galères et leurs solutions !

\end{document}
